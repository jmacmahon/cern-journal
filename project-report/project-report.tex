% !TEX TS-program = xelatex
% !TEX encoding = UTF-8

\documentclass[a4paper,11pt]{article} % use larger type; default would be 10pt
\usepackage{../skills-analysis/default}

\ifxetex
\setmainfont{Droid Sans}
\setmonofont[Scale=0.9]{Droid Sans Mono}
\linespread{1.3}
\fi

\usepgfplotslibrary{fillbetween}

\title{Project Report --- Lumberjack}
\author{Joe MacMahon}
\date{\today}

\begin{document}
\maketitle

\section{Introduction}
\label{sec:introduction}
% TODO dot dot dot
As a part of my placement at CERN, I developed a software library called
Lumberjack over the course of several months.  This will be a critical
evaluation of that project, including [...], and also some surrounding context
of my placement itself.

\subsection{CERN Profile}
\label{sec:cern}
CERN, the European Organisation for Nuclear Research is a world-famous research
facility for particle physics located on the French-Swiss border, just outside
Geneva.  It was founded in 1954 (recently celebrating its 60th anniversary) and
currently has 21 member states from which it receives funding, and since that
time has been responsible for some of the world's most important achievements
in particle physics.

Such a large scientific operation obviously requires a substantial computing
infrastructure to support it, and as such CERN has made significant
contributions to the world of computing.  It is famous as the place where Tim
Berners-Lee created the first incarnation of the World Wide Web in 1991, and
more recently it has been known for developments in grid computing with the LHC
Computing Grid.

In terms of management structure, CERN is composed of a three-layer heirarchy:
there are eight departments\footnote{Beams; Engineering; Finance, Procurement
  and Knowledge Transfer; General Infrastructure Services; Human Resources;
  Information Technology; Physics and Technology}, each of which is divided
into a number of groups, and the groups are divided into sections.  My section,
for example, is IT-CIS-DLS, which means the IT department, Collaboration \&
Information Services (CIS) group, Digital Library Servces (DLS) section.
Ultimately the head of the organisation is the CERN council, which is made up
of delegates from each member state, and the day-to-day top-tier management is
undertaken by the Director-General, appointed for five years by the council.

% TODO talk about tech students

After a shutdown period of more than a year, CERN has just started the second
run of its principle particle accelerator, the Large Hadron Collider, colliding
particles at energies higher than ever previously observed.  This will mean
lots of publication and productive research activity over the coming months and
years and will continue to set the pace of global particle physics research.

\section{Context}
\label{sec:context}
My technical student project at centres on the CERN Document Server, which is a
service to manage papers, reports, press releases, bulletins and multimedia
created at CERN.  It also manages the library's loan system and journal
subscriptions, and provides access to peer-reviewed articles in high-energy
physics to CERN members.  The software that runs CDS is called Invenio, and
also powers a family of other sites as well, for example Inspire and Zenodo.

As with many other large services, a system for gathering and visualising usage
data is important.  In Invenio version 1 this is achieved by logging event data
to MySQL and then generating graphs serverside viewable by an administrator,
however this approach doesn't scale well when a service grows to the size of
CDS.  So, my project is to implement a new system for usage data and analytics
based on Elasticsearch, which is a distributed data store and search engine.

When Elasticsearch is used for storing event-based data, it can be integrated
closely with a web front-end called Kibana, which is used to generate
visualisations.

\section{Proposal}
\label{sec:proposal}

\section{Implementation}
\label{sec:implementation}

\subsection{Setbacks}
\label{sec:setbacks}

\section{Useful Links}
\label{sec:references}

\begin{description*}
  \item[CERN] \url{http://cern.ch}
  \item[CERN Document Server] \url{http://cds.cern.ch/}
  \item[Elasticsearch] \url{https://www.elastic.co/products/elasticsearch}
  \item[Inspire] \url{http://inspirehep.net}
  \item[Kibana] \url{https://www.elastic.co/products/kibana}
  \item[Lumberjack] \url{https://github.com/jmacmahon/lumberjack}
  \item[Zenodo] \url{http://zenodo.org}
\end{description*}

\end{document}
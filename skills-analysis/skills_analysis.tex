% !TEX TS-program = xelatex
% !TEX encoding = UTF-8

\documentclass[a4paper,11pt]{article} % use larger type; default would be 10pt
\usepackage{default}

\ifxetex
\setmainfont{Droid Sans}
\setmonofont[Scale=0.9]{Droid Sans Mono}
\linespread{1.3}
\fi

\usepgfplotslibrary{fillbetween}

\title{Industrial Experience Skills Analysis}
\author{Joe MacMahon}
\date{\today}

\begin{document}
\maketitle

\section{Introduction}
\label{sec:intro}

\section{Technical Skills}
\label{sec:technical}

\subsection{Elasticsearch}
\label{sec:technical.elasticsearch}

Elasticsearch is a distributed data store and powerful search engine, that we use to store log data from various sources.  It operates as a service running on a cluster of servers communicating with each other, and provides a powerful query language for searching and analyzing data.

Our use-case for Elasticsearch is storing log data from CDS\footnote{The CERN Document Server, \url{https://cds.cern.ch/}}, which is a repository for all the papers, reports, and multimedia produced by CERN or its experiments, and manages the CERN library including book loans and journal subscriptions.  The data we collect comes principally from the software that runs CDS (Invenio\footnote{Copyright CERN, GPL.  \url{http://invenio-software.org/}}), and also from our Apache \texttt{access.log} file.

Some examples of events that we collect:
\begin{itemize*}
\item Loan requests
\item Page views
\item Media views
\item Media downloads
\end{itemize*}

I learned how to install, set up and administrate an Elasticsearch cluster using standard Unix tools and an administration interface called Kopf\footnote{Copyright lmenezes, MIT licence.  \url{https://github.com/lmenezes/elasticsearch-kopf}}.  With regards to installation and configuration, the machines we used as nodes run SLC6\footnote{CERN Scientific Linux, \url{http://linux.web.cern.ch/linux/scientific.shtml}}, based on RHEL\footnote{Red Hat Enterprise Linux, \url{http://www.redhat.com/en/technologies/linux-platforms/enterprise-linux}}, which meant that I gained experience in using an RPM-based GNU/Linux system, and in RHEL in particular.  With regards to administration and set-up, I learned how to write complex queries in Elasticsearch's JSON-based query language, and how to manage efficient data storage and indexing within the cluster.

We encountered a problem in that Elasticsearch does not support encrypting the HTTP channel for the REST API it uses to interface with the cluster.  This required some problem-solving, and we decided to implement an Nginx\footnote{A lightweight web server, copyright Nginx Inc., 2-BSD licence.  \url{http://nginx.org/}} front-end on each of the nodes.  Some benchmarking was needed to test the Nginx `layer' and measure the slowdown in throughput with overhead from both Nginx itself and encryption --- for this purpose I used httperf\footnote{Copyright Hewlett-Packard, GPL.  \url{https://github.com/httperf/httperf}}) and ab\footnote{Part of the Apache HTTP Server, copyright The Apache Software Foundation, Apache 2.0 licence.  \url{http://httpd.apache.org/docs/2.4/programs/ab.html}}.  This gave me experience in configuring Nginx, creating SSL keys using OpenSSL, understanding SSL client authentication, and using benchmarking tools effectively.

I also installed and configured two other parts of the typical Elasticsearch stack: Logstash\footnote{Copyright Elasticsearch BV, Apache Licence 2.0.  \url{http://logstash.net}}, a processor and parser for log files, and Kibana\footnote{Copyright Elasticsearch BV, Apache Licence 2.0.  \url{https://www.elastic.co/products/kibana}}, a web-based visualisation system for data stored in Elasticsearch.

In summary, I have gained new skills in RHEL, Elasticsearch, Logstash, Kibana, Kopf, httperf and ab, and deepened my existing skills in Nginx, Unix tools in general, OpenSSL, and GNU/Linux systems administration.

\subsection{Git}
\label{sec:technical.git}
Git is a distributed version control system, which is (usually) used for collective development of a software codebase.  It allows you to manage the history of the codebase, and maintain many branches at once.  A service such as GitHub\footnote{\url{https://github.com/}} can be used to provide a central Git repository along with a variety of other useful features, for example integrated running of test suites with Travis CI\footnote{\url{https://travis-ci.org/}}, automated compilation of documentation using ReadTheDocs\footnote{\url{https://readthedocs.org/}}, and in-house issue tracking and management.

We use a Git repository, hosted on GitHub, to hold the main codebase of Invenio, which allows us to track the different versions of Invenio (\texttt{legacy}, \texttt{maint-1.2}, \texttt{maint-2.0}, \texttt{master}, etc.) and to work on development collaboratively.  We also take advantage of Travis, ReadTheDocs and GitHub's issue management system and our main GitHub repository can be found at \url{http://github.com/inveniosoftware/invenio}.  Although I have used Git and GitHub before for small projects, this is the first time I have explored its more powerful features.

In future I will use Git effectively and powerfully in more of my personal projects, and I can transfer these skills to other (distributed) version control systems in case I work collaboratively on software in those.  If I ever have to do any more software development, it is highly unlikely that it won't use some kind of DVCS to manage the codebase.

\subsection{Software}
\label{sec:technical.software}
More generally, I have gained experience of working on a software project in a team environment.  Previously I have worked on software either as part of university coursework, as a personal project, or as a small contributor to free and open-source software, which has meant that until now I didn't have to focus so much on areas that affect larger software projects.  In developing on Invenio, and writing my own library, Lumberjack\footnote{Copyright CERN, GPL.  \url{http://github.com/jmacmahon/lumberjack}}, I've applied techniques of test-driven development, good quality documentation and well-structured, modularised architecture.  (Of course, these are all good practices which I try to put in place in my other projects, but developing on Invenio  gave me some deeper insight into their usefulness.)

%TODO: more here

\section{Transferable Skills}
\label{sec:transferable}

\subsection{Communication}
\label{sec:transferable.communication}
Good communication skills are vital to ensure a well-operating team at work.  It's important to be able to effectively put a point across in a meeting environment, or just in an informal discussion with a colleague or supervisor; equally important is to be able to listen and understand other people's concerns and input in the same scenarios.

In my case, this is a particularly valuable skill to have, since all of my colleagues are EFL speakers.  While English (and to a lesser extent, French) is used as a lingua franca at CERN, native speakers are relatively rare.  This gave me a new perspective on the importance of using a clear vocabulary, speaking clearly and slowly (but avoiding being patronising), and good phrasing to get ideas across well.  It also requires good judgement in terms of when it is appropriate to correct others' language --- generally when what they're saying can't be understood, or of course if they specifically ask.

In our team we have weekly progress meetings on Thursdays, in which we are expected to give an overview of the work we have done in the previous week.  It's useful to get an idea of what everybody's job is, what things we're working on as a team outside of my specific project, and occasionally offer (or receive) advice or comments on certain aspects.  Additionally, when our team is using agile methodologies, we have a daily scrum meeting in which we assign tasks (using Jira\footnote{\url{https://www.atlassian.com/software/jira/#!}}) and feed back on completed or in-progress tasks.  Both of these have allowed me to develop good summarisation technique and the ability to liaise effectively with other members of the team.

\subsection{Organisation}
\label{sec:transferable.organisation}

- discuss the skill
- describe situations
- what can you do now
- how might I approach in future?

\end{document}